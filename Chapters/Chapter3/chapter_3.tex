% !TeX encoding = UTF-8
% !TeX spellcheck = en_GB
% !TeX root = ../../Thesis.tex
\chapter{Chapter 3}\label{chap:chapter_3}
Tikz sketches and diagrams...


%\begin{figure}[H]
%	\centering	
%	\begin{tikzpicture}[scale=0.55]
%		% !TeX encoding = UTF-8
% !TeX spellcheck = en_US
% !TeX root = ../../Thesis.tex
% set lentghs
\newcommand{\levelLength}{7.0}
\newcommand{\levelLengthLong}{3.0}
\newcommand{\levelConnect}{0.8}
\newcommand{\levelGap}{1.3}
\newcommand{\excitedStateHigh}{21.0}
\newcommand{\excitedStateLow}{11.5}
\newcommand{\labelLength}{2.0}
\newcommand{\arrowGap}{0.5}

% ground state
\draw[level] (0,0) node[left]{$5^2\text{S}_{1/2}$} -- (\levelLength,0);
% F=1
\draw[connect] (\levelLength,0)  -- (\levelLength + \levelConnect,-4.3*\levelGap);
\draw[level]   (\levelLength + \levelConnect, -4.3*\levelGap)  -- (\levelLength + \levelLengthLong + \levelConnect, -4.3*\levelGap) 
node[right, scale=0.9]{$F=1 \quad g_\mathrm{F}=-\frac{1}{2} \quad \left(\SI{-0.70}{\mega \hertz \per \gauss}\right)$}; 
% F=2
\draw[connect] (\levelLength,0) -- (\levelLength + \levelConnect, 2.6*\levelGap);
\draw[level]   (\levelLength + \levelConnect, 2.6*\levelGap)  -- (\levelLength + \levelLengthLong + \levelConnect, 2.6*\levelGap) 
node[right, scale=0.9]{$F=2 \quad g_\mathrm{F}=\frac{1}{2} \quad \left(\SI{0.70}{\mega \hertz \per \gauss}\right)$}; 

% excited state J=1/2
\draw[level] (4.0,\excitedStateLow) node[left]{$5^2\text{P}_{1/2}$} -- (\levelLength,\excitedStateLow);
% F=1
\draw[connect] (\levelLength,\excitedStateLow) -- (\levelLength + \levelConnect, \excitedStateLow - 5.1*\levelGap);
\draw[level]   (\levelLength + \levelConnect, \excitedStateLow - 5.1*\levelGap) 
-- (\levelLength + \levelLengthLong + \levelConnect, \excitedStateLow - 5.1*\levelGap) 
node[right, scale=0.9]{$F=1 \quad g_\mathrm{F}=-\frac{1}{6} \quad \left(\SI{-0.23}{\mega \hertz \per \gauss}\right)$};
% F=2
\draw[connect] (\levelLength,\excitedStateLow) -- (\levelLength + \levelConnect, \excitedStateLow + 3.1*\levelGap);
\draw[level]   (\levelLength + \levelConnect, \excitedStateLow + 3.1*\levelGap) 
-- (\levelLength + \levelLengthLong + \levelConnect, \excitedStateLow + 3.1*\levelGap) 
node[right, scale=0.9]{$F=2 \quad g_\mathrm{F}=\frac{1}{6} \quad \left(\SI{0.23}{\mega \hertz \per \gauss}\right)$};

% excited state J=3/2
\draw[level] (0,\excitedStateHigh) node[left]{$5^2\text{P}_{3/2}$} -- (\levelLength,\excitedStateHigh);
% F=0
\draw[connect] (\levelLength,\excitedStateHigh)  -- (\levelLength + \levelConnect, \excitedStateHigh - 3.0*\levelGap);
\draw[level]   (\levelLength + \levelConnect, \excitedStateHigh - 3.0*\levelGap) 
-- (\levelLength + \levelLengthLong + \levelConnect, \excitedStateHigh - 3.0*\levelGap) node[right, scale=0.9]{$F=0$};
% F=1
\draw[connect] (\levelLength,\excitedStateHigh)  -- (\levelLength + \levelConnect, \excitedStateHigh - 2.3*\levelGap);
\draw[level]   (\levelLength + \levelConnect, \excitedStateHigh - 2.3*\levelGap) 
-- (\levelLength + \levelLengthLong + \levelConnect, \excitedStateHigh - 2.3*\levelGap) 
node[right, scale=0.9]{$F=1 \quad g_\mathrm{F}=\frac{2}{3} \quad \left(\SI{0.93}{\mega \hertz \per \gauss}\right)$};
% F=2
\draw[connect] (\levelLength,\excitedStateHigh)  -- (\levelLength + \levelConnect, \excitedStateHigh - 0.7*\levelGap);
\draw[level]   (\levelLength + \levelConnect, \excitedStateHigh - 0.7*\levelGap) 
-- (\levelLength + \levelLengthLong + \levelConnect, \excitedStateHigh - 0.7*\levelGap) 
node[right, scale=0.9]{$F=2 \quad g_\mathrm{F}=\frac{2}{3} \quad \left(\SI{0.93}{\mega \hertz \per \gauss}\right)$};
% F=3
\draw[connect] (\levelLength,\excitedStateHigh)  -- (\levelLength + \levelConnect, \excitedStateHigh + 1.9*\levelGap);
\draw[level]   (\levelLength + \levelConnect, \excitedStateHigh + 1.9*\levelGap) 
-- (\levelLength + \levelLengthLong + \levelConnect, \excitedStateHigh + 1.9*\levelGap) 
node[right, scale=0.9]{$F=3 \quad g_\mathrm{F}=\frac{2}{3} \quad \left(\SI{0.93}{\mega \hertz \per \gauss}\right)$};

% D1 line
\draw[thick, black, decoration={discontinuity, amplitude=0.2cm, segment length=0.15cm,	meta-segment length=0.5cm}, decorate] (0.8*\levelLength,0) 
-- node[right, scale=0.8, label={[rotate=-45]left:{$\SI{794.979}{\nano \meter}$\ \ }}]{} (0.8*\levelLength, \excitedStateLow);
\draw[thick, black, <-, >=stealth',shorten >=1pt] (0.8*\levelLength, 0) -- (0.8*\levelLength, 3,0);
\draw[thick, black, ->, >=stealth',shorten >=1pt] (0.8*\levelLength, \excitedStateLow - 3.0) -- (0.8*\levelLength, \excitedStateLow);

% D2 line
\draw[thick, black, decoration={discontinuity, amplitude=0.2cm, segment length=0.15cm,	meta-segment length=0.5cm}, decorate] (0.1*\levelLength,0) 
-- node[left, scale=0.8, label={[rotate=-45]left:{$\SI{780.241}{\nano \meter}$}}]{} (0.1*\levelLength, \excitedStateHigh);
\draw[thick, black, <-, >=stealth',shorten >=1pt] (0.1*\levelLength, 0) -- (0.1*\levelLength, 3,0);
\draw[thick, black, ->, >=stealth',shorten >=1pt] (0.1*\levelLength, \excitedStateHigh - 3.0) -- (0.1*\levelLength, \excitedStateHigh);
%	\end{tikzpicture}
%	\caption[\Rb D1,D2 line]{\Rb hyperfine structure of the D1 and D2 line with their respective $g_\mathrm{F}$ factors and linear \person{Zeeman} splitting coefficients. The excited states are 10 times magnified in relation to the ground state.}
%	\label{fig:Rb_hyperfine_structure}
%\end{figure}

%\begin{figure}[htb]
%	\centering
%	\resizebox{\textwidth}{!}{
%		\begin{tikzpicture}[use optics, scale=1.0]
%		% !TeX encoding = UTF-8
% !TeX spellcheck = en_US
% !TeX root = ../../Diplomarbeit_Thomas_Weigner.tex
% set lentghs
\newcommand{\freeLaserShort}{0.5}
\newcommand{\freeLaser}{0.8}
\newcommand{\freeLaserLong}{1.2}
\newcommand{\increment}{=\freeLaserShort of}
\newcommand{\cellHeight}{1.5cm}
\newcommand{\cellRatio}{2.6}
\newcommand{\cellLength}{\cellHeight * \cellRatio}

\node[laser] (LASER) at (0, 0) {Laser};
\draw[red, ->-]  (LASER.aperture east) -- +(\freeLaserShort, 0);
\node[thick optics element, right=\freeLaserShort of LASER.aperture east, object height=1.2cm, object aspect ratio=2.0, label=center:OIso] (OIso) {};
\draw[red, ->-] (OIso.east) -- +(\freeLaserShort, 0);
\node[polarizer, right=\freeLaserShort of OIso.east, label=above:$\frac{\lambda}{2}$] (POL1) {};
\draw[red, ->-]  (POL1.east) -- +(\freeLaserShort, 0);
\node[beam splitter, right=\freeLaserShort of POL1.east, label=above:PBS] (PBS) {};
\draw[red, ->-]  (PBS.south) -- +(0, -\freeLaserLong);
\draw[red, ->-={at=0.35}] (PBS.east) -- +(\freeLaser, 0);
\draw[red, -<-={at=0.65}] (PBS.east) -- +(\freeLaser, 0);
\node[lens, lens type=diverging, right=\freeLaser of PBS.east] (L1) {};
\node[lens, lens type=converging, right=\freeLaserLong of L1] (L2) {};
\draw[red, ->-]  (L1) -- ($(L2.lens north)+(0,-0.2)$) coordinate (upperBeam1);
\draw[red, ->-]  (L1) -- ($(L2.lens south)+(0,+0.2)$) coordinate (lowerBeam1);
\draw[red, ->-={at=0.6}]  (upperBeam1) -- +(\freeLaserShort, 0);
\draw[red, ->-={at=0.6}]  (lowerBeam1) -- +(\freeLaserShort, 0);
\node[thick optics element, right=\freeLaserShort of L2, object height=\cellHeight, object aspect ratio=\cellRatio, label=above:{Vapor Cell}] (VAPCELL) {};
\draw[red] ($(lowerBeam1) + (\freeLaserShort, 0)$) -- +(\cellLength, 0);
\draw[red] ($(upperBeam1) + (\freeLaserShort, 0)$) -- +(\cellLength, 0);
\draw[red, ->-] ($(lowerBeam1) + (\freeLaserShort, 0) + (\cellLength, 0)$) -- +(\freeLaserShort,0) coordinate (lowerBeam2);
\draw[red, ->-] ($(upperBeam1) + (\freeLaserShort, 0) + (\cellLength, 0)$) -- +(\freeLaserShort,0) coordinate (upperBeam2);
\node[diaphragm, right=\freeLaserShort of VAPCELL, slit height = 0.36, object height=2.2cm, label=below:Aperture] (APER) {};
\draw[red, ->-] (APER.slit north) -- +(\freeLaserShort, 0) coordinate (lowerBeam4);
\draw[red, ->-] (APER.slit south) -- +(\freeLaserShort, 0) coordinate (upperBeam4);
\draw[red, ->-] (APER.slit north) -- ($(APER.slit north) + (-2*\freeLaserShort, 0) + (-\cellLength, 0)$) coordinate (upperBeam3);
\draw[red, ->-] (APER.slit south) -- ($(APER.slit south) + (-2*\freeLaserShort, 0) + (-\cellLength, 0)$) coordinate (lowerBeam3);
\draw[red, ->-={at=0.25}] (lowerBeam3) -- (L1);
\draw[red, ->-={at=0.25}] (upperBeam3) -- (L1);
\node[polarizer, right=\freeLaserShort of APER, label=above:$\frac{\lambda}{2}$] (POL2) {};
\draw[red, ->-={at=0.35}]  ($(lowerBeam4) + (POL2.east) - (POL2.west)$) -- +(\freeLaser, 0);
\draw[red, -<-={at=0.65}]  ($(lowerBeam4) + (POL2.east) - (POL2.west)$) -- +(\freeLaser, 0);
\draw[red, ->-={at=0.35}]  ($(upperBeam4) + (POL2.east) - (POL2.west)$) -- +(\freeLaser, 0);
\draw[red, -<-={at=0.65}]  ($(upperBeam4) + (POL2.east) - (POL2.west)$) -- +(\freeLaser, 0);
\node[mirror, right=\freeLaser of POL2] (MIRROR) {};
\node[generic sensor, below=\freeLaserLong of PBS, rotate=-90, right, label=center:{PD}] (PHOTOD) {};
%		\end{tikzpicture}
%	}
%	\caption[DFS setup]{Effective optical setup for a \person{Doppler} free spectroscopy.}
%	\label{fig:dfs}
%\end{figure}

%\begin{figure}[htb]
%	\noindent
%	\centering
%	\begin{subfigure}[t]{0.49\textwidth}
%		\centering
%		\begin{tikzpicture}[use optics, scale=1.0]
%		% !TeX encoding = UTF-8
% !TeX spellcheck = en_US
% !TeX root = ../../Thesis.tex
% set lentghs
\newcommand{\freeLaserShort}{0.5}
\newcommand{\freeLaser}{0.8}
\newcommand{\freeLaserLong}{2}
\newcommand{\increment}{=\freeLaserShort of}

%\node[laser, scale=1.5] (laserDiode) at (0, 0) {\footnotesize LD};
%\draw[rubidiumColor, thick, ->-]   ($(laserDiode.aperture east) + (0,-0.09)$) -- +(\freeLaserLong, 0);
%\draw[rubidiumColor, thick, -<-]   ($(laserDiode.aperture east) + (0,0.09)$) -- +(\freeLaserLong, 0);

%\draw[](-2.8,-1) rectangle (-0.2,1);
%---laser diode
\draw[fill=PdopedColor, PdopedColor](-3,1) rectangle (0,0.15);
\node[black] at (-1.5,0.6) {P-doped};
\draw[fill=NdopedColor, NdopedColor](-3,-1) rectangle (0,-0.15);
\node[black] at (-1.5,-0.6) {N-doped};
\draw[fill=ActiveZoneColor, ActiveZoneColor] (-3,-0.15) rectangle (0,0.15);
\node[black] at (-1.5,0) {Active zone};
\draw[<-, very thick] (-1.5,1) -- (-1.5,1.4) node[right] {I};
\draw[->, very thick] (-1.5,-1) -- (-1.5,-1.4);
%\draw[](0.2,-1) rectangle (0,1);
%\draw[](-3,-1) rectangle (-3.2,1);

%\draw[rubidiumColor, thick, ->-]   (0, 0.09) -- +(\freeLaserLong, 0);
%\draw[rubidiumColor, thick, -<-]   (0, -0.09) -- +(\freeLaserLong, 0);
\draw[rubidiumColor, thick, -<-]   (0, 0) -- +(1.5, 0);
\draw[rubidiumColor, thick, ->-]   (0.5, 0) -- +(1.5, 0);
\draw[rubidiumColor, thick, ->-]   (2, 0) -- +(0, 2);

%---diffrafction grating in Littrow configuration
\coordinate (Grating) at (1.47,-1.2);
\begin{scope}[rotate=63, scale=0.6]
	\foreach \i in {0,...,7}{
		\coordinate (A) at ($(Grating) + (0.5*\i,0)$);
		\coordinate (B) at ($(Grating) + (0.5*\i+0.4,0.2)$);
		\coordinate (C) at ($(Grating) + (0.5*\i+0.5,0)$);
		\draw[fill=DiffractionGratingColor, DiffractionGratingColor] (A) -- (B) -- (C) -- cycle;
	}
	\draw[fill=DiffractionGratingColor, DiffractionGratingColor] (Grating) -- +(0,-0.3) -- ($(Grating) + (8*0.5,-0.3)$) -- +(0,0.3);
	\draw[fill=PiezoColor, PiezoColor] ($(Grating) + (2*0.5,-0.8)$) coordinate (Piezo) rectangle ($(Grating) + (6*0.5,-0.3)$);
%	\foreach \i in {0,...,5}{
%		\draw[thin] ($(Piezo) + (0.2*\i,0)$) -- +(0.08, -0.2);	
%	}
	\draw[fill=DiffractionGratingHandleColor] ($(Grating) + (-0.5,-0.8)$) coordinate (Handle) rectangle ($(Grating) + (8*0.5,-1.2)$);
	\draw[thin] (B) -- +(1.4,0.7)  coordinate (angleLabel1);
	\draw[thin] (C) -- +(1.4,0) coordinate (angleLabel2);
	\draw[thin] (angleLabel1)-- +(0.3,0.15);
	\draw[thin] (angleLabel2) -- +(0.3,0);
	\draw[thin, <->] (angleLabel1) to[bend left] node [pos=0.5] (ThetaLabel) {} (angleLabel2);
	\coordinate (PivotPoint) at ($(Handle) + (0,-0.2)$);
	\coordinate (PiezoLabel) at ($(Piezo) + (1,0.3)$);
\end{scope}
\node[at=(PiezoLabel), inner sep=0.1pt, pin={[pin edge={black, <-}, inner sep=1pt]-60:Piezo}] {};
\node[at=(ThetaLabel), anchor=south, above] {$\Theta_\mathrm{B}$};

%---pivot point
\draw ($(PivotPoint) + (0,-0.1)$) -- +(-0.2, -0.2) coordinate (PivotPointBase)-- +(0.2, -0.2) -- cycle;
\draw ($(PivotPoint) + (0,-0.05)$) circle (0.05);
\foreach \i in {0,...,5}{
	\draw[thin] ($(PivotPointBase) + (0.08*\i,0)$) -- +(-0.03, -0.1);	
}
%		\end{tikzpicture}
%		\subcaption{}
%		\label{fig:laser_types_ecdl}
%	\end{subfigure}
%	\begin{subfigure}[t]{0.49\textwidth}
%		\centering
%		\begin{tikzpicture}[use optics, scale=0.7]
%		% !TeX encoding = UTF-8
% !TeX spellcheck = en_US
% !TeX root = ../../Thesis.tex
\begin{scope}[scale = 0.8, shift ={(4.8, -3)}]
	\draw[ ->]  (0, 0, 0) -- (1, 0);
	\node[label=right:x] at (1, 0) {};
	\draw[->]  (0, 0, 0) -- (0, 1);
	\node[label=above:y] at (0, 1) {};
\end{scope}

\draw[fill=PdopedColor, PdopedColor](-3,2) rectangle (3,0.3);
\node[black] at (0,1.6) {P-doped};
\draw[fill=NdopedColor, NdopedColor](-3,-2) rectangle (3,-0.3);
\node[black] at (0,-1.6) {N-doped};
\draw[fill=ActiveZoneColor, ActiveZoneColor] (-3,-0.3) rectangle (3,0.3);
\node[black] at (0,0) {Active zone};
\draw[rubidiumColor, thick, ->-]  (3,0) -- +(2,0);
%\draw[rubidiumColor, thick, put arrow={arrow’=stealth}] (3,0) -- +(2,0);
\foreach \i in {1,...,12}{
	\draw[fill=GratingColor, GratingColor] (-3.375+0.5*\i, 0.8) rectangle (-3.125+0.5*\i,1);	
}
\draw (-2.75, 0.8) to[dim arrow={label=$\Lambda$}] (-2.25, 0.8);
\node[black, anchor=west] (gratingLabel) at (3.5, 1.2) {Grating};
\draw[<-] (3,0.925) -- (gratingLabel.west);
\draw[<-, very thick] (0,2) -- (0,2.5) node[right] {I};
\draw[->, very thick] (0,-2) -- (0,-2.5);
\node at (-3.5,0) {};
%		\end{tikzpicture}
%		\subcaption{}
%		\label{fig:laser_types_dfb}
%	\end{subfigure}
%	\caption[laser types]{Used semiconductor laser types: (\subref{fig:laser_types_ecdl}) shows a laser diode with an external cavity with a blazed grating in \person{Littrow} configuration. The external cavity length is tune able by piezo. (\subref{fig:laser_types_dfb}) is a sketch of a laser diode with distributed feedback through the grating structure in the semiconductor.}
%	\label{fig:laser_types}
%\end{figure}

%\begin{figure}[htb]
%	\centering
%	\begin{tikzpicture}[scale=0.6]
%	% !TeX encoding = UTF-8
% !TeX spellcheck = en_US
% !TeX root = ../../Diplomarbeit_Thomas_Weigner.tex
% set lentghs
\renewcommand{\levelLength}{2.0}
\renewcommand{\levelLengthLong}{14.0}
\renewcommand{\levelConnect}{0.8}
\renewcommand{\levelGap}{3.5}
\renewcommand{\excitedStateHigh}{16.8}
\renewcommand{\labelLength}{2.0}
\renewcommand{\arrowGap}{0.5}
\newcommand{\offSet}{2}

%scale excited state: \levelGap~100Mhz
%scale ground state:  \levelGap~10Ghz

% ground state
\draw[level] (0,0) node[left]{$5^2\text{S}_{1/2}$} -- (\levelLength,0);
% F=1
\draw[connect] (\levelLength,0)  -- (\levelLength + \levelConnect,-0.43*\levelGap);
\draw[level]   (\levelLength + \levelConnect, -0.43*\levelGap)  -- (\levelLength + \levelLengthLong + \levelConnect, -0.43*\levelGap) 
node[right, scale=0.9]{$F=1$}; 
% F=2
\draw[connect] (\levelLength,0) -- (\levelLength + \levelConnect, 0.256*\levelGap);
\draw[level]   (\levelLength + \levelConnect, 0.256*\levelGap)  -- (\levelLength + \levelLengthLong + \levelConnect, 0.256*\levelGap) 
node[right, scale=0.9]{$F=2$}; 
% gap 1-2 ground
\draw[HelpLineColor, decoration={discontinuity, amplitude=0.2cm, segment length=0.15cm,	meta-segment length=0.5cm}, decorate] (\levelLength + \levelConnect + 0.98*\levelLengthLong, -0.43*\levelGap) -- node[right, HelpLineColor]{\footnotesize \quad $\SI{6.8346}{\giga \hertz}$} (\levelLength + \levelConnect + 0.98*\levelLengthLong, 0.256*\levelGap);
\draw[HelpLineColor, <-, >=stealth',shorten >=1pt] (\levelLength + \levelConnect + 0.98*\levelLengthLong, -0.43*\levelGap) -- (\levelLength + \levelConnect + 0.98*\levelLengthLong, -0.43*\levelGap + 1.0);
\draw[HelpLineColor, ->, >=stealth',shorten >=1pt] (\levelLength + \levelConnect + 0.98*\levelLengthLong, 0.256*\levelGap - 1.0) -- (\levelLength + \levelConnect + 0.98*\levelLengthLong, 0.256*\levelGap);

% excited state J=3/2 
\draw[level] (0,\excitedStateHigh) node[left]{$5^2\text{P}_{3/2}$} -- (\levelLength,\excitedStateHigh);
% F=0
\draw[connect] (\levelLength,\excitedStateHigh)  -- (\levelLength + \levelConnect, \excitedStateHigh - 3.0*\levelGap);
\draw[level]   (\levelLength + \levelConnect, \excitedStateHigh - 3.0*\levelGap) 
-- (\levelLength + \levelLengthLong + \levelConnect, \excitedStateHigh - 3.0*\levelGap) node[right, scale=0.9]{$F'=0$};
% F=1
\draw[connect] (\levelLength,\excitedStateHigh)  -- (\levelLength + \levelConnect, \excitedStateHigh - 2.3*\levelGap);
\draw[level]   (\levelLength + \levelConnect, \excitedStateHigh - 2.3*\levelGap) 
-- (\levelLength + \levelLengthLong + \levelConnect, \excitedStateHigh - 2.3*\levelGap) node[right, scale=0.9]{$F'=1$};
% F=2
\draw[connect] (\levelLength,\excitedStateHigh)  -- (\levelLength + \levelConnect, \excitedStateHigh - 0.7*\levelGap);
\draw[level]   (\levelLength + \levelConnect, \excitedStateHigh - 0.7*\levelGap) 
-- (\levelLength + \levelLengthLong + \levelConnect, \excitedStateHigh - 0.7*\levelGap) node[right, scale=0.9]{$F'=2$};
% F=3
\draw[connect] (\levelLength,\excitedStateHigh)  -- (\levelLength + \levelConnect, \excitedStateHigh + 1.94*\levelGap);
\draw[level]   (\levelLength + \levelConnect, \excitedStateHigh + 1.94*\levelGap) 
-- (\levelLength + \levelLengthLong + \levelConnect, \excitedStateHigh + 1.94*\levelGap) node[right, scale=0.9]{$F'=3$};
% COF=1,3
\draw[level, dashed, CrossOverColor, thick]   (\levelLength + \levelConnect, \excitedStateHigh  - 0.12*\levelGap) 
-- (\levelLength + \levelLengthLong + \levelConnect, \excitedStateHigh - 0.12*\levelGap) node[right, CrossOverColor]{$COF'=1,3$};
% COF=1,2
\draw[level, dashed, CrossOverColor, thick]   (\levelLength + \levelConnect, \excitedStateHigh - 1.5*\levelGap) 
-- (\levelLength + \levelLengthLong + \levelConnect, \excitedStateHigh - 1.5*\levelGap) node[right, CrossOverColor]{$COF'=1,2$};


% gap 0-1 excited
\draw[energyGap, HelpLineColor] (\levelLength + \levelConnect + 0.98*\levelLengthLong, \excitedStateHigh - 3.0*\levelGap) 
-- node[right, HelpLineColor]{\footnotesize $\SI{72.2}{\mega \hertz}$} (\levelLength + \levelConnect + 0.98*\levelLengthLong, \excitedStateHigh - 2.3*\levelGap);
% gap 1-2 excited
\draw[energyGap, HelpLineColor] (\levelLength + \levelConnect + 0.98*\levelLengthLong, \excitedStateHigh - 2.3*\levelGap) 
-- node[right, HelpLineColor, pos=0.25]{\footnotesize$\SI{157.0}{\mega \hertz}$} (\levelLength + \levelConnect + 0.98*\levelLengthLong, \excitedStateHigh - 0.7*\levelGap);
% gap 2-3 excited
\draw[energyGap, HelpLineColor] (\levelLength + \levelConnect + 0.98*\levelLengthLong, \excitedStateHigh - 0.7*\levelGap) -- 
node[right, HelpLineColor]{\footnotesize$\SI{266.7}{\mega \hertz}$} (\levelLength + \levelConnect + 0.98*\levelLengthLong, \excitedStateHigh + 1.94*\levelGap);

% master laser
\draw[transitionLaser, MasterLaserColor, thick] (\levelLength + \levelConnect + 0.02*\levelLengthLong + \offSet,  0.256*\levelGap) 
-- (\levelLength + \levelConnect + 0.02*\levelLengthLong + \offSet, \excitedStateHigh - 0.12*\levelGap);
\node[right, MasterLaserColor, at={(\levelLength + \levelConnect + 0.02*\levelLengthLong + \offSet,  0.256*\levelGap)}, rotate=90, anchor=south west]{\footnotesize Master Laser};

% FO Lock
\draw[transitionLaser, FOLockColor] (\levelLength + \levelConnect + 0.02*\levelLengthLong + \offSet,  \excitedStateHigh - 0.12*\levelGap) 
-- node[right, FOLockColor, pos=0.5, rotate=90, anchor= north, label={[rotate=90, anchor=south]above:{\scriptsize FO Lock}}]{\scriptsize  $\SI{+139}{\mega \hertz}$} (\levelLength + \levelConnect + 0.02*\levelLengthLong + \offSet, \excitedStateHigh + 1.27*\levelGap);
\draw[level, dashed, FOLockColor, thick] (\levelLength + \levelConnect - 0.1*\levelLengthLong + \offSet, \excitedStateHigh + 1.27*\levelGap) 
-- (\levelLength + \levelConnect + 0.35*\levelLengthLong + \offSet, \excitedStateHigh + 1.27*\levelGap);
\draw[transitionLaser, FOShiftColor, thin] (\levelLength + \levelConnect - 0.08*\levelLengthLong + \offSet, \excitedStateHigh + 1.27*\levelGap) -- (\levelLength + \levelConnect - 0.08*\levelLengthLong + \offSet, \excitedStateHigh + 1.44*\levelGap);
\draw[transitionLaser, FOShiftColor, thin] (\levelLength + \levelConnect - 0.08*\levelLengthLong + \offSet, \excitedStateHigh + 1.27*\levelGap) -- (\levelLength + \levelConnect - 0.08*\levelLengthLong + \offSet, \excitedStateHigh + 1.1*\levelGap);

% Cooler AOM
\draw[transitionLaser, AOMColor] (\levelLength + \levelConnect + 0.02*\levelLengthLong + \offSet, \excitedStateHigh + 1.27*\levelGap) 
-- node[right, AOMColor, pos=0.5, rotate=90, label={[rotate=90, anchor=south]above:{\scriptsize Co. AOM}}, anchor= north]{\scriptsize  $+\SI{62.5}{\mega \hertz}$} (\levelLength + \levelConnect + 0.02*\levelLengthLong + \offSet, \excitedStateHigh + 1.8*\levelGap);
\draw[level, dashed, AOMColor, thin] (\levelLength + \levelConnect + \offSet, \excitedStateHigh + 1.8*\levelGap) -- (\levelLength + \levelConnect + 0.5*\levelLengthLong + \offSet, \excitedStateHigh + 1.8*\levelGap);

% Cooling
\draw[transitionLaser, ->|, CoolingColor, thick] (\levelLength + \levelConnect + 0.1*\levelLengthLong + \offSet,  0.256*\levelGap) 
-- (\levelLength + \levelConnect + 0.1*\levelLengthLong + \offSet, \excitedStateHigh + 1.75*\levelGap);
\node[right, CoolingColor, at={(\levelLength + \levelConnect + 0.1*\levelLengthLong + \offSet,  0.256*\levelGap)}, rotate=90, anchor=south west]{\footnotesize Cooling};
\draw[transitionLaser, FOShiftColor, thin] (\levelLength + \levelConnect + 0.1*\levelLengthLong + \offSet, \excitedStateHigh + 1.8*\levelGap) -- (\levelLength + \levelConnect + 0.1*\levelLengthLong + \offSet, \excitedStateHigh + 1.75*\levelGap);
\draw[energyGap, DetuningColor ,thin] (\levelLength + \levelConnect + 0.115*\levelLengthLong + \offSet, \excitedStateHigh + 1.75*\levelGap) 
-- (\levelLength + \levelConnect + 0.115*\levelLengthLong + \offSet, \excitedStateHigh + 1.94*\levelGap) node[above, DetuningColor, rotate=20, anchor=south west]{\footnotesize $\delta=-\SI{19}{\mega \hertz}$};

% Molasses
\draw[transitionLaser, ->|, MolassesColor, thick] (\levelLength + \levelConnect + 0.2*\levelLengthLong + \offSet,  0.256*\levelGap) 
-- (\levelLength + \levelConnect + 0.2*\levelLengthLong + \offSet, \excitedStateHigh + 1.18*\levelGap);
\node[right, MolassesColor, at={(\levelLength + \levelConnect + 0.2*\levelLengthLong + \offSet,  0.256*\levelGap)}, rotate=90, anchor=south west]{\footnotesize Molasses};
\draw[transitionLaser, FOShiftColor, thin] (\levelLength + \levelConnect + 0.2*\levelLengthLong + \offSet, \excitedStateHigh + 1.8*\levelGap) -- (\levelLength + \levelConnect + 0.2*\levelLengthLong + \offSet, \excitedStateHigh + 1.18*\levelGap);
\draw[energyGap, DetuningColor ,thin] (\levelLength + \levelConnect + 0.215*\levelLengthLong + \offSet, \excitedStateHigh + 1.18*\levelGap) 
-- (\levelLength + \levelConnect + 0.215*\levelLengthLong + \offSet, \excitedStateHigh + 1.94*\levelGap) node[above, DetuningColor, rotate=20, anchor=south west]{\footnotesize $\delta=-\SI{76}{\mega \hertz}$};

% Pumping AOM
\draw[transitionLaser, AOMColor] (\levelLength + \levelConnect + 0.3*\levelLengthLong + \offSet, \excitedStateHigh + 1.27*\levelGap) 
-- node[right, AOMColor, pos=0.5, rotate=90, label={[rotate=90, anchor=south]above:{\scriptsize Pump AOM}}, anchor= north]{\scriptsize  $2 \cdot +\SI{80}{\mega \hertz}$} (\levelLength + \levelConnect + 0.3*\levelLengthLong + \offSet, \excitedStateHigh - 0.33*\levelGap);

% Pumping
\draw[transitionLaser, ->|, PumpingColor, thick] (\levelLength + \levelConnect + 0.3*\levelLengthLong + \offSet,  0.256*\levelGap) 
-- (\levelLength + \levelConnect + 0.3*\levelLengthLong + \offSet, \excitedStateHigh - 0.7*\levelGap);
\node[right, PumpingColor, at={(\levelLength + \levelConnect + 0.3*\levelLengthLong + \offSet,  0.256*\levelGap)}, rotate=90, anchor=south west]{\footnotesize Optical Pumping};
\draw[transitionLaser, FOShiftColor, thin] (\levelLength + \levelConnect + 0.3*\levelLengthLong + \offSet, \excitedStateHigh - 0.33*\levelGap) -- (\levelLength + \levelConnect + 0.3*\levelLengthLong + \offSet, \excitedStateHigh - 0.7*\levelGap);

% Imaging
\draw[transitionLaser, ->|, ImagingColor, thick] (\levelLength + \levelConnect + 0.4*\levelLengthLong + \offSet,  0.256*\levelGap) 
-- (\levelLength + \levelConnect + 0.4*\levelLengthLong + \offSet, \excitedStateHigh + 1.94*\levelGap);
\node[right, ImagingColor, at={(\levelLength + \levelConnect + 0.4*\levelLengthLong + \offSet,  0.256*\levelGap)}, rotate=90, anchor=south west]{\footnotesize Imaging};
\draw[transitionLaser, FOShiftColor, thin] (\levelLength + \levelConnect + 0.42*\levelLengthLong + \offSet, \excitedStateHigh + 1.8*\levelGap) -- (\levelLength + \levelConnect + 0.42*\levelLengthLong + \offSet, \excitedStateHigh + 1.94*\levelGap);

% Repump laser
\draw[transitionLaser, RepumperLaserColor, thick] (\levelLength + \levelConnect + 0.8*\levelLengthLong,  -0.43*\levelGap) 
-- (\levelLength + \levelConnect + 0.8*\levelLengthLong, \excitedStateHigh - 1.5*\levelGap);
\node[right, RepumperLaserColor, at={(\levelLength + \levelConnect + 0.8*\levelLengthLong,  0.256*\levelGap)}, rotate=90, anchor=south west]{\footnotesize Repump Laser};

% RepumperAOM
\draw[transitionLaser, AOMColor] (\levelLength + \levelConnect + 0.8*\levelLengthLong,  \excitedStateHigh - 1.5*\levelGap) 
-- node[right, AOMColor, pos=0.5, rotate=90, label={[rotate=90, anchor=south]above:{\scriptsize Rep. AOM}}, anchor= north]{\scriptsize  $+\SI{78.5}{\mega \hertz}$} (\levelLength + \levelConnect + 0.8*\levelLengthLong, \excitedStateHigh - 0.7*\levelGap);

% Repumping
\draw[transitionLaser, ->|, RepumperColor, thick] (\levelLength + \levelConnect + 0.9*\levelLengthLong,  -0.43*\levelGap) 
-- (\levelLength + \levelConnect + 0.9*\levelLengthLong, \excitedStateHigh - 0.7*\levelGap);
\node[right, RepumperColor, at={(\levelLength + \levelConnect + 0.9*\levelLengthLong,  0.256*\levelGap)}, rotate=90, anchor=south west]{\footnotesize Repump};

% D2 line
\draw[thick, black, decoration={discontinuity, amplitude=0.2cm, segment length=0.15cm,	meta-segment length=0.5cm}, decorate] (0.3*\levelLength,0) 
-- node[left, scale=0.8, label=left:{$\SI{780.241}{\nano \meter}$}]{} (0.3*\levelLength, \excitedStateHigh);
\draw[thick, black, <-, >=stealth',shorten >=1pt] (0.3*\levelLength, 0) -- (0.3*\levelLength, 3,0);
\draw[thick, black, ->, >=stealth',shorten >=1pt] (0.3*\levelLength, \excitedStateHigh - 3.0) -- (0.3*\levelLength, \excitedStateHigh);
%	\end{tikzpicture}
%	\caption[laser transitions]{Laser transitions used in the experiment on the \Rb D2-line. The master laser is locked to the $F=1-3$ crossover peak. The TA seeding laser gets FO locked to it and thereby, shifted by $\SI{139}{\mega \hertz}$. After being amplified by the tapered amplifier, it leaves the laser box. The cooling beam going into the MOT-chamber is shifted by an AOM\nomenclature[A-AOM]{AOM}{acusto optical modulator} $+\SI{62.5}{\mega \hertz}$. By changing the voltage of the VCO in the FO lock, the main beam frequency gets shifted. This way all the necessary shifts, indicated by thin pink arrows can be reached. The repumper laser is locked to the $F=1-2$ crossover peak and then shifted with an AOM by $+\SI{78.5}{\mega \hertz}$ to the $F=1 \rightarrow F'=2$ transition.}
%	\label{fig:laser_trasnistions}
%\end{figure}


%\begin{subfigure}[t]{0.98\textwidth}
%	\centering
%	\begin{tikzpicture}[scale=0.8, use optics]
%	% !TeX encoding = UTF-8
% !TeX spellcheck = en_GB
% !TeX root = ../../Diplomarbeit_Thomas_Weigner.tex
\node[generic sensor, anchor=aperture west, rotate=-12] (OutCop) at (0, 0) {};
\node[mirror, Ggreen, rotate=185, object height=1.2cm] (M1) at (-5,1.0) {};
\node[mirror, rotate=5, object height=1.2cm] (M2) at (1.5,2) {};
%chip mount
\begin{scope}[scale =0.5, shift={(-18,4.5)}]
	\draw[fill=ChipMountColor, ChipMountColor] (-2, 4) rectangle (2,3);
	\draw[fill=ChipMountColor, ChipMountColor] (-0.8, 3) rectangle (0.8,1);
	\draw[fill=ChipMountColor, ChipMountColor] (-2, 1) rectangle (2,0);
	\draw[fill=rubidiumColor, rubidiumColor] (0,-0.5) circle (0.2);
	\coordinate (cloud) at (0,-0.5);
%	\node[black] at (0,3.5) {\small chip mount};
%	\foreach \i in {1,...,10}{
%		\draw[black, thick] (-2.4+0.4*\i,4) --(-2.5+0.4*\i,4.4);	
%	}
%	\draw[black, ultra thick] (-2,4) -- (2,4);
\end{scope}
%beam paths
\draw[DirectImgColor, ->-]  ($(OutCop)+(-0.5,-0.1)$) -- ($(M1)+(0,-0.1)$);
\draw[DirectImgColor, ->-]  ($(M1)+(0,-0.1)$) -- (M2);
\draw[DirectImgColor, ->-]  (M2) -- (cloud);
\draw[ReflectImgColor, ->-]  ($(OutCop) + (-0.5,+0.1)$) -- (cloud);
\draw[DirectImgColor, ->-]  (cloud) -- +(-1.5,0);
\draw[ReflectImgColor, ->-]  (cloud) -- +(-1.5,-0.5);
%	\end{tikzpicture}
%	\subcaption{}
%	\label{fig:reflective_imaging_setup}
%\end{subfigure}
%
%
%\begin{subfigure}[t]{0.34\textwidth}
%	\centering
%	\begin{tikzpicture}[scale=0.78, use optics]
%	% !TeX encoding = UTF-8
% !TeX spellcheck = en_GB
% !TeX root = ../../Diplomarbeit_Thomas_Weigner.tex
%optical elements
\node[generic sensor, anchor=aperture west, rotate=180, scale=1.6] (Camera) at (0, 0) {};
\node[lens,lens type=diverging, Ggreen, object height=1.2cm] (L1) at (1,0) {};
\node[lens,lens type=converging, object height=1.2cm] (L2) at (2.5,0) {};
\node[mirror, rotate=-45, object height=1.2cm] (M1) at (4,0) {};
\node[lens,lens type=diverging, rotate=90, object height=1.2cm] (L3) at (4,5) {};
\draw[HelpLineColor, thick, <->] (2.5,4) -- (2.5,6);
\node[HelpLineColor, anchor=east] at (2.5,5) {translation stage};
%beam paths
\draw[rubidiumColor, -<-]  (L3) -- +(0,2);
\draw[rubidiumColor, ->-]  (L3) -- (M1);
\draw[rubidiumColor, ->-]  (M1) -- (L2);
\draw[rubidiumColor, ->-]  (L2) -- (L1);
\draw[rubidiumColor, ->-]  (L1) -- (Camera);
%	\end{tikzpicture}
%	\subcaption{}
%	\label{fig:main_imaging_setup}
%\end{subfigure}